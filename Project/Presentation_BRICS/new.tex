\documentclass[8pt,hideothersubsections]{beamer}
\newcommand\Fontvi{\fontsize{9}{11}\selectfont}
\newcommand\Fontcap{\fontsize{6}{8}\selectfont}
\newcommand\Fontot{\fontsize{7.5}{9}\selectfont}
%\bibliographystyle{./References/agu} 
\usepackage{amsmath}
\usepackage{amssymb}
\usepackage{graphicx}
\usepackage{url}
\usepackage{xfrac}
\usepackage{esvect}
\usepackage{varwidth} 
\usepackage{physics}
\usepackage{tensor}
\usepackage{microtype}

\setlength\abovecaptionskip{-5pt}

%%%%%%%%%%%%%%%%%%%%%%Multiple equations on the same line%%%%%%%%%%%%%%%%%%%%
\usepackage{multicol}
%%%%%%%%%%%%%%%%%%%%%%%%%%%%%%%%%%%%%%%%%
\newcommand{\driv}[2]{\frac{d #1}{d #2}}
\newcommand{\con}{\frac{8\pi}{3}G}
\newcommand{\brac}[1]{\left(#1\right)}
\newcommand{\bracc}[1]{\left[#1\right]}
%%%%%%%%%%%%%%%%%%%%%%%%%%%%%%%%%%%%%%%%%%                                        

\usepackage{multicol}
\usepackage{natbib}
\usepackage{amsmath}
%\usetheme{Madrid}
\usetheme{Rochester}
\setbeamertemplate{navigation symbols}{}
\setbeamertemplate{frame numbering}{fraction}
\usecolortheme[named=violet]{structure}
\setbeamercolor{background canvas}{bg=white}

\setbeamercovered{transparent=5}
%\usebeamerfont{title}\insertsectionhead\par%

\makeatletter
\setbeamertemplate{sidebar canvas right}[vertical shading][top=black,bottom=violet]
\setbeamertemplate{footline}
{
\leavevmode%
\hbox{%
\begin{beamercolorbox}[wd=0.5\paperwidth, ht=2.25ex,dp=1ex, center]{author in head/foot}%
\usebeamerfont{author in head/foot}\insertauthor
\end{beamercolorbox}
\begin{beamercolorbox}[wd=0.5\paperwidth, ht=2.25ex,dp=1ex, right]{title in head/foot}%
\usebeamerfont{title in head/foot}
%\inserttitle \hspace*{0.5em}
\insertframenumber{}/\inserttotalframenumber\hspace*{2ex}
\end{beamercolorbox}}%
\vskip0pt%
}

\makeatother
\title{Solving the Friedmann equation for a Dark Fluid equation of state.}
\subtitle{}
\author{Pieter vd Merwe}
\institute{North-West University Centre for Space Research}
\date{\today}

\begin{document}

\begin{frame}
\titlepage
%\begin{flushright}
%\includegraphics[scale=0.3]{./Images/nwu_logo.jpg}
%\end{flushright}
%\begin{flushleft}
%\includegraphics[scale=0.4]{./Images/NASSP_logo.jpg}
%\end{flushleft}
\begin{figure}[ht]
    \begin{minipage}{0.45\linewidth}
        \centering
        \includegraphics[width=\textwidth]{./Images/NASSP_logo.jpg}
    \end{minipage}
    \begin{minipage}{0.45\linewidth}
        \centering
        \includegraphics[width=\textwidth,height=30pt]{./Images/nwu_logo.jpg}
    \end{minipage}
\end{figure}
\end{frame}

\section{Solving the Friedmann equations for two different dark fluid equations of state}
\begin{frame}
\frametitle{\insertsectionhead}
\begin{itemize}
\item Looking at two different equations of state that aims to parametrize between dark energy and dark matter dominated epochs.\\
\vspace{15pt}
\item Modified Chaplygin gas (MCG):
\begin{equation}\label{eq:MCG}
\begin{split}
P &=A_{2}\rho -\frac{A_{1}}{\rho^{\alpha}},\ \ \ \ \alpha>-1        \\
\end{split}
\end{equation}
\item The PPUDF equation of state:
\begin{equation}\label{eq:UDFEoS}
\begin{split}
P &= P_{a}+P_{b}\brac{z+\frac{z}{1+z}}         \\
\end{split}
\end{equation}
\end{itemize}
\end{frame}
%
%\section{Introduction}
%\subsection{Aim and Motivation}
%\begin{frame}
%\begin{itemize}
%\frametitle{\insertsectionhead}
%\framesubtitle{\insertsubsectionhead}
%
%\item 2 Dark elements\\
%\hspace{2pt}
%
%
%\item Is it possible to describe the cosmological behaviour resulting from these elements using a single dark fluid?\\ 
%
%
%
%\end{itemize}
%\end{frame}
%
%\begin{frame}
%\frametitle{Introduction}
%\begin{itemize}
%\item \textbf{General Relativity:}\\
%Einstein`s field equations:
%\begin{equation}\label{eq:GR}
%\begin{split}
%\tensor{G}{_{\mu\nu}}+\Lambda\tensor{g}{_{\mu\nu}}&=\frac{8\pi G}{c^{4}}\tensor{T}{_{\mu\nu}}\\
%\end{split}
%\end{equation}
%\item \textbf{Hubble`s law:}
%\begin{equation}\label{eq:Hubble}
%\begin{split}
%\nu&=H_{0}r\\
%\end{split}
%\end{equation}
%\item \textbf{Expanding universe:}
%\begin{equation}\label{eq:scale}
%\begin{split}
%r(t) &= a(t)\chi\\
%\end{split}
%\end{equation}
%
%\end{itemize}
%\end{frame}
%
%\section{Friedmann equations}
%\begin{frame}
%\begin{itemize}
%\frametitle{\insertsectionhead}
%
%\item \textbf{Cosmological principle:}\\
%The universe is homogeneous and isotropic \citep{ITC}.
%
%\item \textbf{Fluid equation:}
%\begin{equation}\label{eq:6}
%\begin{split}
%\dot{\rho}+3H\left(1+\omega\right)\rho &= 0,\ \text{with } H=\frac{\dot{a}}{a}\\
%\end{split}
%\end{equation}
%\item \textbf{Friedmann equation:}
%\begin{equation}
%\begin{split}\label{eq:CurvFriedman}
%\dot{a}^{2} &= \frac{8\pi G}{3c^{2}}\rho a^{2}-\kappa\frac{c^{2}}{\chi^{2}}\\
%\end{split}
%\end{equation}
%\item \textbf{Raychaudhuri equation:}
%\begin{equation}\label{eq:RayEq}
%\begin{split}
%\frac{\ddot{a}}{a} &= -\frac{4\pi}{3}G\left(\rho +3P\right)\\
%\end{split}
%\end{equation}
%
%\end{itemize}
%\end{frame}
%
%\section{Concordance model ($\Lambda$CDM-model)}
%\begin{frame}
%\begin{itemize}
%\frametitle{\insertsectionhead}
%\item \textbf{Hot Big Bang}
%\item \textbf{Dark Matter:}
%\begin{itemize}
%\item[$-$] Galaxy rotation\\
%\hspace{2pt}
%\item[$-$] Behaviour of super structure\\
%\end{itemize}
%\item \textbf{Accelerated expansion:}\\
%Observations of the luminosities of type Ia supernovae suggest that the universe is undergoing an accelerated expansion \citep{NPSNe, RMCGAU}, which suggests the existence of a Dark energy element.
%\item \textbf{Assume a perfect fluid equation of state:}
%\begin{equation}\label{eq:PFEoS}
%\begin{split}
%P &= \omega\rho         \\
%\end{split}
%\end{equation}
%\item \textbf{3 Different epochs:}
%\begin{itemize}
%\item[$-$] Radiation ($\omega=\frac{1}{3}$): $\rho=C_{rad}a^{-4}$ \\
%\hspace{2pt}
%\item[$-$] Matter ($\omega=0$): $\rho=C_{dust}a^{-3}$ \\
%\hspace{2pt}
%\item[$-$] Dark Energy ($\omega=-1$): $\rho=C_{DE}$\\
%\end{itemize}
%
%\end{itemize}
%\end{frame}
%
%\section{Chaplygin gas}
%\begin{frame}
%\begin{itemize}
%\frametitle{\insertsectionhead}
%\item Different Chaplygin gas equations of state \citep{kahya2015universe}:
%\begin{itemize}
%\item[$-$] Original Chaplygin gas (OCG):
%\begin{equation}\label{eq:OCG}
%\begin{split}
%P &= -\frac{A_{1}}{\rho}         \\
%\end{split}
%\end{equation}
%\item[$-$] Generalised Chaplygin gas (GCG):
%\begin{equation}\label{eq:GCG}
%\begin{split}
%P &= -\frac{A_{1}}{\rho^{\alpha}},\ \ \ \ \alpha>-1         \\
%\end{split}
%\end{equation}
%\item[$-$] Modified Chaplygin gas (MCG):
%\begin{equation}\label{eq:MCG}
%\begin{split}
%P &=A_{2}\rho -\frac{A_{1}}{\rho^{\alpha}},\ \ \ \ \alpha>-1        \\
%\end{split}
%\end{equation}
%\end{itemize}
%\end{itemize}
%\end{frame}

%
%\subsection{Solution to the fluid equation}
%\begin{frame}
%\begin{itemize}
%\fontsize{8pt}{7.2}\selectfont
%\frametitle{\insertsectionhead}
%\framesubtitle{\insertsubsectionhead}
%
%\item \textbf{Solving the Fluid equation for a MCG equation of state:}
%\begin{equation}\label{eq:FMCGZ}
%\begin{split}
%\rho  &= \bracc{\frac{C_{2}\brac{1+z}^{3\brac{\alpha+1}\brac{1+A_{2}}}+A_{1}}{1+A_{2}}}^{\frac{1}{1+\alpha}} \\
%\end{split}
%\end{equation}
%
%\end{itemize}
%\begin{figure}[h]
%\centering
%\includegraphics[scale=0.45]{Images/ch_rho.jpg}
%\caption{Here we have taken $A_{1}=50$, $A_{2}=C_{2}=1$ and $\alpha=1$. The figure shows energy density $\rho$ vs red-shift $z$.}
%\label{fig:ChRho}
%\end{figure}
%\end{frame}

%\subsection{Hubble parameter for MCG case}
%\begin{frame}
%\frametitle{\insertsectionhead}
%\framesubtitle{\insertsubsectionhead}
%\fontsize{8pt}{7.2}\selectfont
%\begin{itemize}
%\item Dimensionless Hubble parameter $h$
%\begin{equation}\label{eq:zChDimHubbleParm}
%\begin{split}
%h(z) &= \frac{1}{H_{0}}\bracc{A\brac{B_{3}\brac{1+z}^{3\brac{\beta}\brac{B_{1}}}+B_{2}}^{\frac{1}{\beta}} -\kappa F\brac{1+z}^{2}}^{\frac{1}{2}}\\
%\end{split}
%\end{equation}
%\item Fractional energy density $\Omega$
%\begin{equation}\label{eq:ChFracEnDen}
%\begin{split}
%\Omega_{Chap}(z) &\equiv \frac{A}{H_{0}^{2}}\brac{B_{3}\brac{1+z}^{3\brac{\beta}\brac{B_{1}}}+B_{2}}^{\frac{1}{\beta}} \\
%\Omega_{\kappa}(z)&\equiv -\frac{\kappa F}{H_{0}^{2}}\brac{1+z}^{2}\\
%\end{split}
%\end{equation}
%\end{itemize}
%
%\begin{figure}[ht]
%    \begin{minipage}{0.49\linewidth}
%        \centering
%        \includegraphics[width=\textwidth]{./Images/ch_H.jpg}
%		\caption{The figure shows the dimensionless Hubble parameter $h$ vs red-shift $z$.}
%		\label{fig:ChH}
%    \end{minipage}
%    \begin{minipage}{0.49\linewidth}
%        \centering
%        \includegraphics[width=\textwidth]{./Images/ch_Om.jpg}
%		\caption{The figure shows fractional energy density $\Omega$ vs red-shift $z$.}
%		\label{fig:ChFracEnDen}
%    \end{minipage}
%\end{figure}
%
%\end{frame}

\subsection{Energy density and acceleration of a for MCG case}
\begin{frame}
\frametitle{\insertsubsectionhead}
\fontsize{8pt}{7.2}\selectfont
\begin{itemize}
\item \textbf{Solving the Fluid equation for a MCG equation of state:}
\begin{equation}\label{eq:FMCGZ}
\begin{split}
\rho  &= \bracc{\frac{C_{2}\brac{1+z}^{3\brac{\alpha+1}\brac{1+A_{2}}}+A_{1}}{1+A_{2}}}^{\frac{1}{1+\alpha}} \\
\end{split}
\end{equation}
\fontsize{8pt}{7.2}\selectfont
\item \textbf{Deceleration parameter} $q\equiv-\frac{\ddot{a}a}{\dot{a}^{2}}$
\fontsize{6pt}{7.2}\selectfont
\begin{equation}\label{eq:ChModDecelZ}
\begin{split}
q &= \frac{\frac{A}{2}\brac{\brac{3B_{1}-2}\brac{B_{3}\brac{1+z}^{3B_{1}\beta}+B_{2}}^{\frac{1}{\beta}}-3B_{1}B_{2}\brac{B_{3}\brac{1+z}^{3\beta B_{1}}+B_{2}}^{\frac{1-\beta}{\beta}}}}{A\brac{B_{3}\brac{1+z}^{3\brac{\beta}\brac{B_{1}}}+B_{2}}^{\frac{1}{\beta}} -\kappa F\brac{1+z}^{2}}
\end{split}
\end{equation}
\end{itemize}

\begin{figure}[ht]
    \begin{minipage}{0.49\linewidth}
        \centering
		\includegraphics[width=\textwidth]{Images/ch_rho.jpg}
		\caption{Here we have taken $A_{1}=50$, $A_{2}=C_{2}=1$ and $					\alpha=1$. The figure shows energy density $\rho$ vs red-shift $z$.}
		\label{fig:ChRho}
    \end{minipage}
    \begin{minipage}{0.49\linewidth}
        \centering
        \includegraphics[width=\textwidth]{./Images/ch_q.jpg}
		\caption{The figure shows deceleration parameter $q$ vs red-shift $z$.}
		\label{fig:Chq}
    \end{minipage}
\end{figure}
\end{frame}


%\subsection{Solving the Friedmann equation for the MCG equation of state}
%\begin{frame}
%\begin{itemize}
%\frametitle{\insertsectionhead}
%\framesubtitle{\insertsubsectionhead}
%\fontsize{7pt}{7.2}\selectfont
%\item Assuming a $\kappa=0$, we have:
%\fontsize{6pt}{7.2}\selectfont
%\begin{equation}\label{eq:FmEqMCGSol}
%\begin{split}
%\brac{t-t_{0}}&=\frac{2}{3A^{\frac{1}{2}}B_{2}^{\frac{1}{2\beta}}B_{1}}\brac{\frac{B_{3}}{B_{2}}a^{-3B_{1}\beta}+1}^{-\frac{1}{2\beta}}\\
%+\frac{1}{2\beta+1}&\bracc{\brac{\frac{B_{3}}{B_{2}}a^{-3B_{1}\beta}+1}^{-1-\frac{1}{2\beta}}\ _{2}F_{1}\brac{1,1+\frac{1}{2\beta};2+\frac{1}{2\beta};\brac{\frac{B_{3}}{B_{2}}a^{-3B_{1}\beta}+1}^{-1}}}\\
%\end{split}
%\end{equation}
%\end{itemize}
%\begin{figure}[H]
%\centering
%\includegraphics[scale=0.45]{Images/a_ch.jpg}
%\caption{The figure shows scale factor $a$ vs time $t$.}
%%\caption{Here equation (\ref{eq:NumSolMCG}) has been integrated numerically for $a'=0$ to $a'=1$ in incremental steps using the Gauss quadrature method. For this we have set $\frac{8\pi G}{3c^{2}}=\frac{c^{2}}{\chi^{2}}=1$ in order to investigate the behaviour of the model without looking at the physical constraints, and taken all of the free parameters to be the same as in the previous case. }
%\label{fig:ChScale}
%\end{figure}
%\end{frame}


%\section{Pressure-Parametrized Unified Dark Fluid (PPUDF)}
%\subsection{Solution to the fluid equation}
{
%\setbeamerfont{frametitle}{size=\small}
%\begin{frame}
%\begin{itemize}
%\frametitle{\insertsectionhead}
%\fontsize{7pt}{7.2}\selectfont
%\item \textbf{The PPUDF equation of state \citep{wang2017new}:}
%\begin{equation}\label{eq:UDFEoS}
%\begin{split}
%P &= P_{a}+P_{b}\brac{z+\frac{z}{1+z}}         \\
%\end{split}
%\end{equation}
%\item \textbf{Solving the Fluid equation for a PPUDF equation of state:}
%\begin{equation}\label{eq:UDFFluidSolZ}
%\begin{split}
%\rho&= -P_{a}+\frac{3}{4}P_{b}\bracc{\brac{1+z}^{-1}-2\brac{1+z}}+C\brac{1+z}^{3} \\
%\end{split}
%\end{equation}
%
%\end{itemize}
%\begin{figure}[h]
%\centering
%\includegraphics[scale=0.45]{Images/UDF_rho.jpg}
%\caption{The figure shows energy density $\rho$ vs red-shift $z$.}
%\label{fig:UDFRho}
%\end{figure}
%\end{frame}

%\subsection{Hubble parameter for PPUDF case}
%\begin{frame}
%\frametitle{\insertsectionhead}
%\framesubtitle{\insertsubsectionhead}
%\fontsize{8pt}{7.2}\selectfont
%\begin{itemize}
%\item Dimensionless Hubble parameter $h$
%\fontsize{6pt}{7.2}\selectfont
%\begin{equation}\label{eq:UDFDimh}
%\begin{split}
%h &= \frac{1}{H_{0}}\bracc{A\brac{-P_{a}+\frac{3}{4}P_{b}\bracc{\brac{1+z}^{-1}-2\brac{1+z}}+C\brac{1+z}^{3}} -\kappa F \brac{1+z}^{2}}^{\frac{1}{2}}\\
%\end{split}
%\end{equation}
%\fontsize{8pt}{7.2}\selectfont
%\item Fractional energy density $\Omega$
%\fontsize{6pt}{7.2}\selectfont
%\begin{equation}\label{eq:UDFOmega}
%\begin{split}
%\Omega_{PPUDF}(z) &\equiv \frac{A}{H_{0}^{2}}\brac{-P_{a}+\frac{3}{4}P_{b}\bracc{\brac{1+z}^{-1}-2\brac{1+z}}+C\brac{1+z}^{3}}      \\
%\Omega_{\kappa}(z)&\equiv -\frac{\kappa F}{H_{0}^{2}}\brac{1+z}^{2}\\
%\end{split}
%\end{equation}
%\end{itemize}
%
%\begin{figure}[ht]
%    \begin{minipage}{0.49\linewidth}
%        \centering
%        \includegraphics[width=\textwidth]{./Images/UDF_H.jpg}
%		\caption{The figure shows the dimensionless Hubble parameter $h$ vs red-shift $z$.}
%		%\caption{Here the Parameters have, again, been taken as in the previous cases and 			we have also set $H_{0}=1$.}
%		\label{fig:UDFH}
%    \end{minipage}
%    \begin{minipage}{0.49\linewidth}
%        \centering
%        \includegraphics[width=\textwidth]{./Images/UDF_Om.jpg}
%		\caption{The figure shows fractional energy density $\Omega$ vs red-shift $z$.}
%		%\caption{Taking the various parameters as in the previous cases, the figure shows 			the fractional energy density of the Chaplygin gas fluid as well as for the 				different curvature cases.It is important to point out here that the fractional 			energy density of the different curvatures are non-linear, but increases much 				slower than the Chaplygin gas fractional energy density for the chosen parameters.}
%		\label{fig:UDFFracEnDen}
%    \end{minipage}
%\end{figure}
%
%\end{frame}

\subsection{Energy and acceleration of a for PPUDF case}
\begin{frame}
\frametitle{\insertsubsectionhead}
\fontsize{8pt}{7.2}\selectfont
\begin{itemize}
\item \textbf{Solving the Fluid equation for a PPUDF equation of state:}
\begin{equation}\label{eq:UDFFluidSolZ}
\begin{split}
\rho&= -P_{a}+\frac{3}{4}P_{b}\bracc{\brac{1+z}^{-1}-2\brac{1+z}}+C\brac{1+z}^{3} \\
\end{split}
\end{equation}
\fontsize{8pt}{7.2}\selectfont
\item \textbf{Deceleration parameter} $q\equiv-\frac{\ddot{a}a}{\dot{a}^{2}}$
\fontsize{6pt}{7.2}\selectfont
\begin{equation}\label{eq:UDFq}
\begin{split}
q &= \frac{A\bracc{2P_{a}-\frac{3}{2}P_{b}\brac{\frac{3}{2}\brac{1+z}^{-1}+\brac{1+z}}+C\brac{1+z}^{3}}}{2A\brac{-P_{a}+\frac{3}{4}P_{b}\bracc{\brac{1+z}^{-1}-2\brac{1+z}}+C\brac{1+z}^{3}} -\kappa F \brac{1+z}^{2}}       \\
\end{split}
\end{equation}
\end{itemize}

\begin{figure}[ht]
    \begin{minipage}{0.49\linewidth}
		\centering
		\includegraphics[width=\textwidth]{Images/UDF_rho.jpg}
		\caption{The figure shows energy density $\rho$ vs red-shift $z$.}
		\label{fig:UDFRho}
    \end{minipage}
    \begin{minipage}{0.49\linewidth}
    	\vspace{10pt}
        \centering
        \includegraphics[width=\textwidth]{./Images/UDF_q.jpg}
		\caption{The figure shows deceleration parameter $q$ vs red-shift $z$.}
		\label{fig:Chq}
    \end{minipage}
\end{figure}
\end{frame}
}


\section{Conclusions and Acknowledgements}
\setbeamerfont{frametitle}{size=\Huge}
\begin{frame}
\begin{itemize}
\frametitle{\insertsectionhead}

\item Both Chaplygin gas and the PPUDF equations of state result in behaviour for the energy densities and acceleration that corresponds with the Concordance model for dust dominated epochs and Dark energy dominated epochs.\\
\hspace{2pt}

\item It is possible to unify the epochs of both dark elements into a single dark fluid epoch by parametrizing the equation of state.\\
\hspace{2pt}

\item Shortcomings of the Chaplygin gas and PPUDF models.\\
\hspace{2pt}


\item Future work on this would include constraining the free parameters with observation. \\
\hspace{2pt}

\item I just wish to thank Dr. Abebe and Dr. Mongwane for their inputs and supervision as well as NASSP for the funding.

\end{itemize}
\end{frame}

%\section{Acknowledgements}
%\begin{frame}
%\begin{itemize}
%\frametitle{\insertsectionhead}
%
%\item I just wish to thank Dr. Abebe and Dr. Mongwane for their inputs and supervision as well as NASSP for the funding.
%\end{itemize}
%\end{frame}

%
%\begin{frame}
%\bibliographystyle{unsrt}                               
%\bibliography{./References/References} 
%\end{frame}



\end{document}