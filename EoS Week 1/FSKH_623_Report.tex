\documentclass[a4paper, 11pt]{FSKH_623_Report} 

\title{Equation of state}                                                                      
\problemnumber{}                                             
\author{Pieter vd Merwe}                                         
\studentnumber{25957937}                                           
\date{\today}  

\usepackage{amsmath}
\usepackage{amssymb}
\usepackage{graphicx}
\usepackage{url}
\usepackage{xfrac}
\usepackage{esvect}
\usepackage{varwidth} 
%%%%%%%%%%%%%%%%%%%%%%%%%%%%%%%%%%%%%%%%%%
\newcommand{\driv}[2]{\frac{d #1}{d #2}}
%%%%%%%%%%%%%%%%%%%%%%%%%%%%%%%%%%%%%%%%%%                                        

\begin{document}

\maketitle

\pagenumbering{gobble}                                             
\tableofcontents                                                  
\pagebreak
\pagenumbering{arabic}

\section{Equation of State from Newtonian Mechanics}

From the first law of thermodynamics we have that  
\begin{equation}
\begin{split}
dE &= -PdV + dQ\\
\end{split}
\end{equation}
With $dE$ the change in energy content, $P$ the pressure, $dV$ the change in volume and $dQ$ the change in heat.
Considering an Isotropic and homogeneous universe \citep{notes4},the expansion of the universe will be adiabatic with the Pressure $P$ constant and there will be no change in the heat ($Q$) content of the universe since there is nowhere for the heat to come from \citep{notes4}. From this we can then conclude that the $dQ$ term in equation 1 will be $dQ=0$ and reduces to.
\begin{equation}
\begin{split}
dE &= -PdV \\
\end{split}
\end{equation}
this impies that
\begin{equation}
\begin{split}
E(t) &= -PV(t)\\
\Rightarrow \dot{E} &= \frac{d\left(-PV\right)}{dt}\\
&= -\driv{P}{t}V-P\driv{V}{t}\\
&= -P\driv{V}{t}             \\
\end{split}
\end{equation}
since $P$ is a constant.
We have that for M the total mass and c the speed of light 
\begin{equation}
\begin{split}
V &= \frac{4\pi}{3}r^{3}\\
E &= Mc^{2}\\
&=\frac{4\pi}{3}c^{2}r^{3}\rho\\
\end{split}
\end{equation}
where $\rho$ is the mass density as a function of time.
If we now consider a expanding universe where the relative motion of galaxies are not due to intrinsic peculiar velocities, but to the expansion of the universe then we can write $r$ as $r=a(t)\chi$ where $\chi$ is the actual distance between the two points (galaxies) and $a(t)$ the expansion parameter defined so that $a(t)=1$ at the present time. Equations 4 then becomes
\begin{equation}
\begin{split}
V &= \frac{4\pi}{3}a^{3}\chi^{3}\\
E &= Mc^{2}\\
&=\frac{4\pi}{3}c^{2}a^{3}\chi^{3}\rho\\
\end{split}
\end{equation}
From this then follows that
\begin{equation}
\begin{split}
\dot{V} &= 4\pi\chi^{3}a^{2}\dot{a}\\
\end{split}
\end{equation}
and
\begin{equation}
\begin{split}
\dot{E} &= \frac{4\pi\chi^{3} c^{2}}{3}\left(3a^{2}\dot{a}\rho+\dot{\rho}a^{3}\right)\\
\end{split}
\end{equation}
Substituting this back into equation 3 yields
\begin{equation}
\begin{split}
\frac{4\pi\chi^{3} c^{2}}{3}\left(3a^{2}\dot{a}\rho+\dot{\rho}a^{3}\right) &= -P4\pi\chi^{3}a^{2}\dot{a}\\
\Rightarrow \dot{\rho}+3\frac{\dot{a}}{a}\rho+3P\frac{\dot{a}}{a} &= 0\\
\Rightarrow \dot{\rho}+3\frac{\dot{a}}{a}\left(\rho+P\right) &= 0\\
\Rightarrow \dot{\rho}+3H\left(\rho+P\right) &= 0\\
\end{split}
\end{equation}
Where $H$ is the Hubble parameter.
If we now let $P=\rho\omega$, where $\omega$ can be a function of time, in such a way that P remains a constant.
then equation 8 becomes
\begin{equation}
\begin{split}
\dot{\rho}+3H\left(\rho+\rho\omega\right) &= 0\\
\dot{\rho}+3H\left(1+\omega\right)\rho &= 0\\
\end{split}
\end{equation}
We can solve this ODE 
\begin{equation}
\begin{split}
\dot{\rho}+3H\left(1+\omega\right)\rho &= 0\\
\Rightarrow \frac{\dot{\rho}}{\rho} &= -3H\left(1+\omega\right)\\
\Rightarrow \frac{\dot{\rho}}{\rho} &= -3\left(1+\omega\right)\frac{\dot{a}}{a}\\
\Rightarrow ln(\rho) &= -3\left(1+\omega\right)ln(a) + k\\
\Rightarrow \rho &= Ca^{-3\left(1+\omega\right)}
\end{split}
\end{equation}
with $C$ and $k$ constants.
For matter $\omega=0$, for radiation $\omega=\frac{1}{3}$ and for the cosmological constant $\omega=-1$ \citep{notes4}. From this then follows that for matter 
\begin{equation}
\begin{split}
\rho &= Ca^{-3}\\
\end{split}
\end{equation}
for radiation
\begin{equation}
\begin{split}
\rho &= Ca^{-4}\\
\end{split}
\end{equation}
and for the cosmological constant
\begin{equation}
\begin{split}
\rho &= C\\
\end{split}
\end{equation}

















\citep{notes4}
\bibliographystyle{./References/agu}                               
\bibliography{./References/References} 
\pagenumbering{gobble}                            
\pagebreak

%\lstinputlisting[language=Python, firstline=1, lastline=27, caption=Python example]{Densities.py}

\end{document}  